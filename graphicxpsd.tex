%#!rm -f tigerpsdfmt4* && lualatex -shell-escape graphicxpsd
\documentclass[luatex]{article}
\usepackage{shortvrb}\MakeShortVerb{\|}
\usepackage{graphicx}
\usepackage{graphicxpsd}
\title{\textsf{graphicxpsd} Package}
\author{Munehiro Yamamoto}
\date{2018/01/26 v1.0}
\begin{document}
\maketitle
\begin{abstract}
This package provides Adobe Photoshop Data format (PSD) support 
for \textsf{graphicx} package
with \texttt{sips} (Darwin/macOS)/\texttt{convert} (ImageMagick) command.
\end{abstract}

\section{Motivation}
\texttt{graphicx} package supports already many graphics image formats as bellow. 
\begin{itemize}
\item non-vector formats: jpg, png, bmp, and so on
\item PostScript-style formats: eps, ps
\item PDF-style formats: pdf, ai
\end{itemize}
However, it currently does not support Adobe Photoshop Data format (PSD). 

Against that, we developed the \textsf{graphicxpsd} package 
to support PSD format via PSD-to-PDF conversion 
with the following image converters.
\begin{itemize}
\item \texttt{sips}: pre-installed command in Darwin/macOS
\item \texttt{convert}: bundled command in ImageMagick
\end{itemize}

\section{Loading \textsf{graphicxpsd} Package}

Load \textsf{graphicxpsd} package after loading \textsf{graphicx} package.

\begin{quote}
\begin{verbatim}
\usepackage{graphicx}
\usepackage[<options>]{graphicxpsd}
\end{verbatim}
\end{quote}

The list of available options is the following.
\begin{itemize}
\item |dvipdfmx|, |xetex|, |pdftex|, |luatex|: supported driver options; 
You can also give specific driver option from global option.

\item |sips| (default), |imagemagick|: supported converter
\end{itemize}

\section{Example}

Typeset the following {\LaTeX} document with LuaTeX.

\begin{quote}
\begin{verbatim}
%#!lualatex -shell-escape
\documentclass[luatex]{article}%%set luatex driver as global option
\usepackage{graphicx}
\usepackage{graphicxpsd}
\begin{document}
\includegraphics{tigerpsdfmt.psd}
\end{document}
\end{verbatim}
\end{quote}

Then, the result is as below.

\includegraphics{tigerpsdfmt.psd}

\end{document}
